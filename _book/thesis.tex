% This is the Oregon State University LaTeX template. To the best of my
% knowledge, most of the work was done by those acknowledged in beavtex.cls.

%%
%% Preamble
%%
% \documentclass{<something>} must begin each LaTeX document
\documentclass[double,12pt]{beavtex}
% Added by CII
\usepackage{graphicx,latexsym}
\usepackage{amsmath}
\usepackage{amssymb,amsthm}
\usepackage{longtable,booktabs,setspace}
\usepackage[hyphens]{url}
\usepackage[colorlinks = true, 
			urlcolor = blue,
			linkcolor = black,
			citecolor = black,
			anchorcolor = black]{hyperref}
\usepackage{lmodern}
\usepackage{float}
\floatplacement{figure}{H}
% End of CII addition
\usepackage{rotating} % Package added to allow the rotation of figures and chart
                      % on a page, {sidewaysfigure} command
\usepackage{tablefootnote} % Packaged added to allow footnotes in the tabular
                           % environment, use \tablefootnote command

% This has to do with a default pandoc thing
% http://tex.stackexchange.com/a/258486/77699
\providecommand{\tightlist}{%
  \setlength{\itemsep}{0pt}\setlength{\parskip}{0pt}}

% Added by CII (Thanks, Hadley!)
% Use ref for internal links
\renewcommand{\hyperref}[2][???]{\autoref{#1}}
\def\chapterautorefname{Chapter}
\def\sectionautorefname{Section}
\def\subsectionautorefname{Subsection}
% End of CII addition

% Added by CII
\usepackage{caption}
\captionsetup{width=5in}
% End of CII addition

\title{An Analysis of Something} % {An Analysis of Something}
\author{Zhian N. Kamvar} % {Joseph A. Student}
\degree{Doctor of Philosophy} % {Master of Science}
\doctype{Dissertation}
\submitdate{January 1, 2013} % {January 1, 2013}
\commencementyear{2017} % {2013}

\department{Botany and Plant Pathology} % {Nuclear Engineering and Radiation Health Physics}

\depttype{Department} % {School}

\depthead{Head} % {Director}

\major{Plant Pathology} % {Radiation Health Physics}

\advisor{Niklaus J. Grünwald} % {Jane R. Professor}

\abstract{This is a \LaTeX template derived from the \beavtex template found on
overleaf
(\url{https://www.overleaf.com/latex/templates/oregon-state-university-thesis-and-dissertation/wnvzcdhqshxf})
and cloned using
\texttt{git\ clone\ https://git.overleaf.com/5973847rnpkdf} I have made
the following changes

\begin{itemize}
\tightlist
\item
  modified \texttt{beavtex.cls} from v1.1 to include support for
  verbatim font styles in R code
\item
  stripped the template of its original content to create a pandoc
  template
\item
  placed this into a bookdown (\url{https://bookdown.org/}) framework
\item
  added packages latexsym, amsmath amssymb,amsthm, longtable, booktabs,
  setspace, url, hyperref, lmodern, caption, and float
\end{itemize}

This is still a WIP, but it seems to work for the moment. Zhian N.
Kamvar 2016-08-25}
\acknowledgements{I would like to acknowledge\ldots{}Lorem ipsum dolor sit amet,
consectetur adipiscing elit. Maecenas vel eros sed mauris porttitor
semper nec a orci. Nullam vestibulum mi nec condimentum posuere.
Pellentesque eget diam id sapien aliquet ullamcorper. Pellentesque
blandit nec lectus ut mollis. Praesent in facilisis justo. Vestibulum
ante ipsum primis in faucibus orci luctus et ultrices posuere cubilia
Curae; Sed eget congue leo, sed consequat libero. In rutrum malesuada
nisi. Vestibulum ante ipsum primis in faucibus orci luctus et ultrices
posuere cubilia Curae; Morbi sollicitudin tortor ut sem facilisis
mollis.}


	\contributors{The following people contributed to this dissertation:

\subsubsection*{Chapter 1}\label{chapter-1}
\addcontentsline{toc}{subsubsection}{Chapter 1}

Jane R. Professor assisted in the design, analysis, and editing of the
manuscript.

\subsubsection*{Chapter 2}\label{chapter-2}
\addcontentsline{toc}{subsubsection}{Chapter 2}

Lisa Simpson developed the initial concept and experimental design for
the study. Ellen Ripley advised and assisted with statistical analysis.
Jane R. Professor assisted in the design, analysis, and editing of the
manuscript.

\subsubsection*{Chapter 3}\label{chapter-3}
\addcontentsline{toc}{subsubsection}{Chapter 3}

Jane R. Professor assisted in the design, analysis, and editing of the
manuscript.

\subsubsection*{Chapter 4}\label{chapter-4}
\addcontentsline{toc}{subsubsection}{Chapter 4}

Jane R. Professor assisted in the design, analysis, and editing of the
manuscript.}


	\dedication{This is for my mother who paved the way.}




\begin{document}

\maketitle
\mainmatter


  \chapter*{Introduction}\label{introduction}
  \addcontentsline{toc}{chapter}{Introduction}
  
  Welcome to the \emph{R Markdown} thesis template. This template is based
  on (and in many places copied directly from) the
  \href{https://www.overleaf.com/latex/templates/oregon-state-university-thesis-and-dissertation/wnvzcdhqshxf}{Oregon
  State University LaTeX template} and wrapped up into Chester Ismay's
  \href{https://github.com/ismayc/thesisdown}{\emph{thesisdown} package},
  but hopefully it will provide a nicer interface for those that have
  never used TeX or LaTeX before. Using \emph{R Markdown} will also allow
  you to easily keep track of your analyses in \textbf{R} chunks of code,
  with the resulting plots and output included as well. The hope is this
  \emph{R Markdown} template gets you in the habit of doing reproducible
  research, which benefits you long-term as a researcher, but also will
  greatly help anyone that is trying to reproduce or build onto your
  results down the road.
  
  Hopefully, you won't have much of a learning period to go through and
  you will reap the benefits of a nicely formatted thesis. The use of
  LaTeX in combination with \emph{Markdown} is more consistent than the
  output of a word processor, much less prone to corruption or crashing,
  and the resulting file is smaller than a Word file. While you may have
  never had problems using Word in the past, your thesis is likely going
  to be about twice as large and complex as anything you've written
  before, taxing Word's capabilities. After working with \emph{Markdown}
  and \textbf{R} together for a few weeks, we are confident this will be
  your reporting style of choice going forward.
  
  \textbf{Why use it?}
  
  \emph{R Markdown} creates a simple and straightforward way to interface
  with the beauty of LaTeX. Packages have been written in \textbf{R} to
  work directly with LaTeX to produce nicely formatting tables and
  paragraphs. In addition to creating a user friendly interface to LaTeX,
  \emph{R Markdown} also allows you to read in your data, to analyze it
  and to visualize it using \textbf{R} functions, and also to provide the
  documentation and commentary on the results of your project. Further, it
  allows for \textbf{R} results to be passed inline to the commentary of
  your results. You'll see more on this later.
  
  \textbf{Who should use it?}
  
  Anyone who needs to use data analysis, math, tables, a lot of figures,
  complex cross-references, or who just cares about the final appearance
  of their document should use \emph{R Markdown}. Of particular use should
  be anyone in the sciences, but the user-friendly nature of
  \emph{Markdown} and its ability to keep track of and easily include
  figures, automatically generate a table of contents, index, references,
  table of figures, etc. should make it of great benefit to nearly anyone
  writing a thesis project.
  
  \section*{Objective}\label{objective}
  \addcontentsline{toc}{section}{Objective}
  
  The purpose of this study is to\ldots{} Lorem Smith \& Jones (1973)
  ipsum dolor sit amet, consectetur adipiscing elit. Sed venenatis nunc
  sapien. Praesent imperdiet nulla eu rutrum venenatis. Fusce rhoncus urna
  a nunc semper, non venenatis lorem tempor. Cras sollicitudin eget velit
  eu venenatis. Mauris imperdiet pretium massa sed dapibus. Nunc ipsum
  ipsum, porttitor ut urna ut, pretium feugiat leo. Nunc magna enim,
  facilisis a porttitor eget, elementum ac turpis. Quisque et gravida
  justo. Etiam vulputate quam at commodo suscipit. Vivamus ut adipiscing
  tortor. Phasellus quis dolor et mi hendrerit sollicitudin.
  
  Cras dapibus congue mauris, et imperdiet magna pellentesque non. Sed
  venenatis adipiscing quam ut placerat. Praesent imperdiet dignissim
  cursus. Phasellus mattis nibh vitae semper pellentesque. Lorem ipsum
  dolor sit amet, consectetur adipiscing elit. Sed dignissim tellus id
  adipiscing tempus. Aenean posuere malesuada rhoncus. Ut quis elit eros.
  
  \section*{Background}\label{background}
  \addcontentsline{toc}{section}{Background}
  
  Lorem ipsum dolor sit amet, consectetur adipiscing elit. Sed venenatis
  nunc sapien. Praesent imperdiet nulla eu rutrum venenatis. Fusce rhoncus
  urna a nunc semper, non venenatis lorem tempor. Cras sollicitudin eget
  velit eu venenatis. Mauris imperdiet pretium massa sed dapibus. Nunc
  ipsum ipsum, porttitor ut urna ut, pretium feugiat leo. Nunc magna enim,
  facilisis a porttitor eget, elementum ac turpis. Quisque et gravida
  justo. Etiam vulputate quam at commodo suscipit. Vivamus ut adipiscing
  tortor. Phasellus quis dolor et mi hendrerit sollicitudin.
  
  Cras dapibus congue mauris, et imperdiet magna pellentesque non. Sed
  venenatis adipiscing quam ut placerat. Praesent imperdiet dignissim
  cursus. Phasellus mattis nibh vitae semper pellentesque. Lorem ipsum
  dolor sit amet, consectetur adipiscing elit. Sed dignissim tellus id
  adipiscing tempus. Aenean posuere malesuada rhoncus. Ut quis elit eros.
  
  \backmatter
  
  \chapter{References}\label{references}
  
  \noindent
  
  \setlength{\parindent}{-0.20in} \setlength{\leftskip}{0.20in}
  \setlength{\parskip}{8pt}
  
  \hypertarget{refs}{}
  \hypertarget{ref-angel2000}{}
  Angel, E. (2000). \emph{Interactive computer graphics : A top-down
  approach with opengl}. Boston, MA: Addison Wesley Longman.
  
  \hypertarget{ref-angel2001}{}
  Angel, E. (2001a). \emph{Batch-file computer graphics : A bottom-up
  approach with quicktime}. Boston, MA: Wesley Addison Longman.
  
  \hypertarget{ref-angel2002a}{}
  Angel, E. (2001b). \emph{Test second book by angel}. Boston, MA: Wesley
  Addison Longman.
  
  \hypertarget{ref-grunwald2011phytophthora}{}
  Grünwald, N. J., Martin, F. N., Larsen, M. M., Sullivan, C. M., Press,
  C. M., Coffey, M. D., \ldots{} Parke, J. L. (2011). Phytophthora-ID.
  org: a sequence-based \emph{Phytophthora} identification tool.
  \emph{Plant Disease}, \emph{95}(3), 337--342.
  
  \hypertarget{ref-smith}{}
  Smith, J., \& Jones, J.-J. (1973). Some kind of title. \emph{Health
  Physics}, \emph{1}, 1000--1010.


\end{document}

